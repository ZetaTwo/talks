\documentclass[10pt, compress]{beamer}

\usetheme{metropolis}

\usepackage{booktabs}
\usepackage[scale=2]{ccicons}

\usepackage{amsmath}

%Icons
\usepackage{fontawesome}

%Syntax highlight
\usepackage{listings,xcolor}
\usepackage{inconsolata}

\definecolor{dkgreen}{rgb}{0,.6,0}
\definecolor{dkblue}{rgb}{0,0,.6}
\definecolor{dkyellow}{cmyk}{0,0,.8,.3}

% Language: PHP
\lstset{
  language        = php,
  basicstyle      = \small\ttfamily,
  keywordstyle    = \color{dkblue},
  stringstyle     = \color{red},
  identifierstyle = \color{dkgreen},
  commentstyle    = \color{gray},
  emph            =[1]{php},
  emphstyle       =[1]\color{black},
  emph            =[2]{if,and,or,else},
  emphstyle       =[2]\color{dkyellow}
}

%\usepgfplotslibrary{dateplot}


\title{GDPR ur ett startup-perspektiv}	
%\subtitle{}
\date{2017-11-14}
\author{Carl Svensson}
\institute{Cloud Confessions 2017}

\begin{document}

\maketitle

\begin{frame}{Om mig}
  
	\begin{columns}
		\begin{column}{0.6\textwidth}  
  
  		\begin{itemize}
		  \item Carl Svensson, 26
		  \item Civilingenjör, teknisk fysik, KTH
		  \item Säkerhetschef, Kry
		  \item \faEnvelope \hskip 2mm carl@kry.se
		  \item \faGlobe \hskip 2mm https://kry.se
		\end{itemize}
		
		\end{column}
		\begin{column}{0.4\textwidth} 
			\begin{center}
			\includegraphics[width=0.4\textwidth]{images/kth.jpg}
			\end{center}
			\vspace{1cm}
			\includegraphics[width=\textwidth]{images/kry_logo.png}
		\end{column}
	\end{columns}
  
\end{frame}

\begin{frame}{Kry - Framtidens sjukvård}
\includegraphics[width=\textwidth]{images/kry_logo.png}
\end{frame}

\begin{frame}{Kry - Framtidens sjukvård}

 \begin{itemize}
  \item Vårdteknikbolag
  \item Videomöten med läkare
  \item Tillgänglig vård för alla
  \item Unga men snabbväxande
  \end{itemize}    

\end{frame}

\begin{frame}{Startup - Utmaningar}

 \begin{itemize}
  \item Få etablerade processer
  \item Kort (gemensam) erfarenhet
  \item Snabbväxande
  \item Patiantdatalagen vs GDPR
  \end{itemize}    

\end{frame}

\begin{frame}{Startup - Möjligheter}

 \begin{itemize}
  \item Ingen process-/tekniskskuld
  \item Ingen långsam byråkrati
  \item Säkerhetstänk från start
  \item Agila
  \end{itemize}    

\end{frame}

\begin{frame}{GDPR hos Kry - Översikt}

 \begin{itemize}
  \item Interna processer
  \item Data science
  \item Spårning av användare
  \item Right to be forgotten
  \end{itemize}    

\end{frame}

\begin{frame}{GDPR hos Kry - Intressenter}

 \begin{itemize}
  \item RnD: CTO, VP of Eng och jag
  \item Legal team
  \item Team leads
  \end{itemize}    

\end{frame}

\begin{frame}{GDPR hos Kry - Interna processer}

 \begin{itemize}
  \item Uppstartsmöten
  \item Kartläggning
  \item Uppföljningsmöten
  \item Visualiserings
  \item Förbättring
  \item Dokumentation
  \end{itemize}    

\end{frame}

\begin{frame}{GDPR hos Kry - Data science}

 \begin{itemize}
  \item Dataklassificering
  \item PII \& PDL-data
  \item Pseudonymiserad
  \item Anonymiserad
  \item Klustring \& differential privacy
  \end{itemize}    

\end{frame}

\begin{frame}{GDPR hos Kry - Tracking}

 \begin{itemize}
  \item Appen, enkelt
  \item Webben, svårt
  \item Tredjepartsleverantörer
  \item Legal team
  \end{itemize}    

\end{frame}

\begin{frame}{GDPR hos Kry - Right to be forgotten}

 \begin{itemize}
  \item PDL vs PII
  \item Avstängda användare
  \item Säkerhetskopior
  \item Tredjepartsintegrationer
  \end{itemize}    

\end{frame}

\begin{frame}{GDPR hos Kry - Framtiden}

 \begin{itemize}
  \item Bli klara med GDPR-projekt
  \item Jobba med leverantörer
  \item Etablera framtida processer \& mindset
  \end{itemize}    

\end{frame}


\begin{frame}[standout]
Frågor?
\end{frame}



\end{document}
